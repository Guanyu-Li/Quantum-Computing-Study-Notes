\chapter{量子查询算法}

传统的计算机基本就是按照下面这个图的形式进行:

$$
\begin{tikzpicture}
% Nodes
\node (输入) at (0,0) [] {输入};
\node (计算) at (3,0) [draw, rectangle, minimum height=1cm, minimum width=2cm, fill=blue!30] {计算};
\node (输出) at (6,0) [] {输出};

% Arrows
\draw[-latex] (输入) -- (计算);
\draw[-latex] (计算) -- (输出);
\end{tikzpicture}
$$

不同于上述这种过程,下面要介绍的查询模型,输入不是直接提供给计算机的.而是以函数的形式提供,计算机通过查询来访问输入.
$$
\begin{tikzpicture}
	% Draw blocks
	\node (input) at (0,0) [draw, rectangle, minimum height=2em, minimum width=2cm, fill=blue!20] {输入};
	\node () at (0,-1)[]{$\cdots$};
	\node (computation) at (0,-2) [draw, rectangle, minimum height=2em, minimum width=2cm, fill=blue!20] {计算机};
	
	% Draw arrows
	\draw[<-] ([xshift=-0.7cm]input.south) -- ([xshift=-0.7cm]computation.north)node[midway, left] {查询};
	\draw[->] ([xshift=-0.4cm]input.south) -- ([xshift=-0.4cm]computation.north);
	\draw[<-] ([xshift=0.4cm]input.south) -- ([xshift=0.4cm]computation.north);
	\draw[->] ([xshift=0.7cm]input.south) -- ([xshift=0.7cm]computation.north);
	% Output arrow
     \draw[->] (computation.east) -- ++(2,0) node[right] {输出};
\end{tikzpicture}
$$

在查询模型的计算中,输入通常由一个oracle或blackbox提供.这两个词语的意思就是说,输入的完整描述对于计算来说是隐藏的,我们无法看见函数的内部结构或理解它的工作机制,唯一的访问方式就是提出问题.


\section{Deutsch算法}

考虑实现这样功能的一个量子电路:
	\begin{Quantikz}
	\centering
	\begin{quantikz}
	\lstick{$\ket{x}$}&\gate[2]{U_{x_0}}&\rstick{$\ket{x}$} \\
	\lstick{$\ket{y}$}&& \rstick{$\ket{y \oplus f(x)}$} 
	\end{quantikz}
	\caption{相位返还}\label{circle1}
\end{Quantikz}


以$f: \{0,1\} \to \{0,1\}$为例,不妨带入$\ket{-}$:
\begin{center}
	\begin{tikzcd}
		\ket{x}\ket{-} \xmapsto{~U_{x_0}~} \ket{x}\otimes\dfrac{1}{\sqrt{2}}(\ket{f(x)}-\overline{\ket{f(x)}})=\ket{x}\otimes (-1)^{f(x)}\ket{-} .
	\end{tikzcd}	
\end{center} 

也就是实现了\begin{equation}\label{back}
\ket{x}\ket{-} \xmapsto ~ (-1)^{f(x)} \ket{x}\ket{-}
\end{equation}

这个过程称为相位返还,这里的$U_{x_0}$实际上就是前文提到的oracle,对于具体问题我们目前不考虑物理上构造它的难度,只是定义了一个理论模型,有助于阐明量子计算的潜在优势而已.

有两个盒子,每个盒子里可能装着一个苹果或一个橙子.我们想知道:这两个盒子是否装着同样类型的水果.考虑映射 $f: \{0,1\} \to \{0,1\}$. $f$可以看作是打开盒子后观察水果类型的操作,那么问题转化为对于映射$f: \{0,1\} \to \{0,1\}$, 判断$f(0)$是否$=f(1)$,这便是Deutsch问题.

显然直接查询两次可以直接回答 $f(0)$是否等于 $f(1)$.但是通过\cref{circle1},考虑将$\ket{x}$换为$\ket{+}$,那么可以构造出$f(1) \otimes f(1)$:
\begin{center}
\begin{quantikz}
	\lstick{$\ket{+}$}&\gate[2]{U_{x_0}}&\rstick{$\ket{+}$} \\
	\lstick{$\ket{-}$}&& \rstick{$\dfrac{(-1)^{f(0)}}{\sqrt{2}}(1+(-1)^{f(0) \oplus f(1)})\ket{-}$} 
\end{quantikz}.
\end{center}
这样就可以通过一次操作来判断 $f(0)$是否等于 $f(1)$,称为Deutsch算法.

该算法的多量子比特版本称为Deutsch-Jozsa算法,即给定函数 $f: \{0,1\}^n \to \{0,1\}$ ,需要满足要么 $f$ 是一个常数函数(即所有输入得到相同输出),要么 $f$ 是一个平衡函数(一半输入得到0,另一半得到1),希望确定是哪一种函数.

事实上,在前文Deutsch问题中,也就是$n=1$的情况下, $f(0)=f(1)$是常数函数, $f(0)\neq f(1)$是平衡函数,现在根据Deutsch问题的经验考虑如下的电路图:
	\begin{Quantikz}
	\centering
	\begin{quantikz}
		\lstick{$\ket{0}$}&\gate{H}&\gate[5]{U_{x_0}}&\gate{H}&\meter{} \\
		\lstick{$\ket{0}$}&\gate{H}&&\gate{H}&\meter{}\\
		\lstick{$\hspace{-0.7cm}\boldsymbol{\vdots}$} \rstick{$\hspace{4.3cm}\boldsymbol{\vdots}$}\\
		\lstick{$\ket{0}$}&\gate{H}&&\gate{H}&\meter{}\\
		\lstick{$\ket{-}$}&&&&
	\end{quantikz}
	\caption{Deutsch-Jozsa算法}
\end{Quantikz}
 
注意到:
\begin{equation*}
	(H\ket{0})^{\otimes n} \ket{-} = \ket{+}^{\otimes n} \ket{-} = \frac{1}{\sqrt{2^n}} \sum_{x \in \{0, 1\}^n} \ket{x} \ket{-} \xmapsto{U_{x_0}} \frac{1}{\sqrt{2^n}} \sum_{x \in \{0, 1\}^n} (-1)^{f(x)} \ket{x} \ket{-} .
\end{equation*}
并且,
\begin{equation}
	H \ket{x} = \frac{\ket{0} + (-1)^x \ket{1}}{\sqrt{2}}, \quad x \in \{0, 1\}.
\end{equation}

代入之后有:
\begin{equation}
	H^{\otimes n} \ket{x} = \bigotimes_{j=1}^n \frac{\ket{0} + (-1)^{x_j} \ket{1}}{\sqrt{2}} 
	= \frac{1}{\sqrt{2^n}} \sum_{y \in \{0, 1\}^n} \prod_{j=1}^n (- 1)^{x_j y_j} \ket{y}.
\end{equation}

其中$x_j,y_j$为第$j$个分量,假设$y_j $要取0,那么前面对应位置就需要是$\ket{0}$做的张量积,如果让这个位置的系数为1,那么等号满足;同样的,要是$y_j$取1,那么这个部位的系数也为$(-1)^{x_j}$,因此从每一位来看等号是成立的,整体上来看等号是成立的.值得注意的是,首先使用$H$门对输入的$\ket{0}$作用,在量子算法中是很常用的操作.

原式转化为,
\begin{equation*}
	\frac{1}{\sqrt{2^n}} \sum_{x \in \{0, 1\}^n} (-1)^{f(x)} \ket{x} \ket{-} \xmapsto{H^{\otimes n}\otimes I} \frac{1}{2^n} \sum_{x \in \{0,1\}^n} (-1)^{f(x)} \sum_{y \in \{0,1\}^n} (-1)^{x \cdot y} \ket{y} \ket{-}
	= \sum_{y \in \{0, 1\}^n} a_y \ket{y} \ket{-}
\end{equation*}
其中:
\begin{equation}
	a_y = \frac{1}{2^n} \sum_{x \in \{0, 1\}^n} (-1)^{f(x) + x \cdot y} .
\end{equation}

如果是常数函数,记为$f(x) = a$,此时
\begin{equation*}
	a_y = \frac{(-1)^a}{2^n} \sum_{x \in \{0, 1\}^n} (-1)^{x \cdot y},~
	a_{0, \ldots, 0} = (-1)^a,~ a_y=0(y \neq 0, \ldots, 0).
\end{equation*}

其余情况为平衡函数,有:
\begin{equation*}	
a_{0, \ldots, 0} = \frac{1}{2^n} \sum_{x \in \{0,1\}^n} (-1)^{f(x)} = \frac{1}{2^n}(2^{n-1}-2^{n-1})=0.
\end{equation*}

综上测量结果为全$\ket{0}$的时候为常数函数,其余情况为平衡函数.

\section{非结构化搜索问题}
\subsection{Grover算法}
接着上一节的例子,现在我们的目标是找到含有橙子的盒子.问题转化为,对于 $ f : \{0, 1\}^n \rightarrow \{0, 1\} $,并且存在一个唯一的标记状态 $ x_0 $ 使得 $ f(x_0) = 1 $,我们希望找到 $ x_0 $.这是典型的非结构化搜索问题,因为我们对于$f$的内在结构(如何从输入得到输出)是不清楚的,无法直接构建一个结构化的查询来寻找$x_0$,而是挨个盒子打开检查,需要经历$N-1$次的穷举.结构化搜索指的,是例如有一个包含运动员信息的数据库,记录包含姓名、国籍、项目和成绩.如果需要找到所有参加“100米跑”且来自“中国”的运动员的记录,根据“项目”和“国籍”字段进行过滤,这就是结构化搜索.

针对上述非结构化搜索问题,考虑使用第一节的相位返还算子$U$,将本节研究的问题带入之后就是
\begin{equation}
\begin{cases}
\ket{x}  \xmapsto ~ \ket{x}, f(x)=0; \\
\ket{x}  \xmapsto ~ -\ket{x}, f(x)=1.
\end{cases}
\end{equation}

设
\begin{equation}
	\vert \psi \rangle = \frac{1}{\sqrt{N}} \sum_{x \in [N]} \vert x \rangle.
\end{equation}
 
 可以改写为:
\begin{equation}
 	\vert \psi \rangle =\dfrac{1}{\sqrt{N}}\ket{\omega}+\sqrt{1-\dfrac{1}{\sqrt{N}}}\ket{\omega^\bot}
 	= \sin(\theta) \vert \omega \rangle + \cos(\theta) \vert \omega^\bot \rangle ,
\end{equation}
 其中:
 $$
 \ket{\omega^\bot}=\frac{1}{\sqrt{N-1}} \sum_{x \in [N],\bra{x_0}\ket{x}=0} \vert x \rangle,~ \theta = \arcsin \frac{1}{\sqrt{N}}.
 $$
 
考虑酉矩阵$ V_\psi =  2 \ket{\psi}\bra{\psi} -I$ ,有
 $$
 \begin{cases}
 	V\ket{\psi} = \ket{\psi}, \\
 	V\ket{\omega} = \dfrac{2}{\sqrt{n}}\ket{\psi}-\ket{\omega}.
 \end{cases}
 $$
 
注意到:
\begin{equation}
 \begin{cases} 
 VU \ket{\omega}=(2 \ket{\psi}\bra{\psi} -I)(-\ket{\omega})=(1-2 \sin^2 \theta )\ket{\omega}+2 \sin \theta \cos \theta \ket{\omega^\bot},\\
 VU \ket{\omega^\bot}=(2 \ket{\psi}\bra{\psi} -I)(\ket{\omega^\bot})=-2 \sin \theta \cos \theta \ket{\omega}+(2 \cos^2 \theta - 1)\ket{\omega^\bot}.
 \end{cases}
\end{equation}
 
为了表示方便可以将这个变换整理成矩阵:
\begin{equation}
\begin{pmatrix}
		1 - 2 \sin^2 \theta &- 2 \sin \theta \cos \theta \\
		2 \sin \theta \cos \theta & 2 \cos^2 \theta - 1
	\end{pmatrix} = \begin{pmatrix}
		\cos 2\theta &- \sin 2\theta \\
		\sin 2\theta & \cos 2\theta
	\end{pmatrix} = \begin{pmatrix}
		\cos \theta &- \sin \theta \\
		\sin \theta & \cos \theta
	\end{pmatrix} ^2
\end{equation}
 
 这也就说明进行一次$G=VU$相当于做两次角度为$\theta$的旋转变换,则$G^k \vert \psi \rangle = \sin((2k+1)\theta) \vert \omega \rangle + \cos((2k+1)\theta) \vert \omega^\bot_0 \rangle .$想让 $ \sin((2k+1)\theta/2) \approx 1 $,需要 $ k \approx \frac{\pi}{2\theta} \approx \frac{\pi}{4} \sqrt{N} $,也就是说Grover 算法可以通过 $ O(\sqrt{N}) $ 次查询解决无结构搜索问题.

接下来考虑量子电路,考虑$Z_{\text{OR}} = 2 \ket{0^n} \bra{0^n} - I$,对两边做H门有:
\begin{equation}
H^{\otimes n} Z_{\text{OR}} H^{\otimes n} = 2 \ket{\psi} \bra{\psi} - I=V
\end{equation}

在这里我们得到了$V$的量子电路,则该算法通过以下电路实现:
	\begin{Quantikz}
	\centering
	\begin{quantikz}
		\lstick{$\ket{0}$}& \gate{H}& \gate[4]{U_{x_0}} \gategroup[4,steps=4,style={dashed,rounded corners,fill=blue!20,inner xsep=2pt},background,label style={label position=below,anchor=north,yshift=-0.2cm}]{G}  & \gate{H} & \gate[4]{Z_{\text{OR}}} & \gate{H} & \meter{} \\
		\lstick{$\ket{0}$} & \gate{H} & \qw           & \gate{H} & \qw                      & \gate{H} & \meter{} \\
		\lstick{$\ket{0}$} & \gate{H} & \qw           & \gate{H} & \qw                      & \gate{H} & \meter{} \\
		\lstick{$\ket{0}$} & \gate{H} & \qw           & \gate{H} & \qw                      & \gate{H} & \meter{}
	\end{quantikz}
	\caption{Grover算法}
\end{Quantikz}
\vspace{1cm}

当$x_0$不唯一时,该方法也同样使用,令$A_0  = \{ x \in \Sigma^n : f(x) = 0 \},~A_1 = \{ x \in \Sigma^n : f(x) = 1 \}$.则有:
\begin{equation}
	\ket{\omega} = \frac{1}{\sqrt{|A_0|}} \sum_{x \in A_0} \ket{x},~\ket{\omega^\bot} = \frac{1}{\sqrt{|A_1|}} \sum_{x \in A_1} \ket{x}.	
\end{equation}

\begin{equation}
	\ket{\psi} = \sqrt{\frac{|A_0|}{N}} \ket{\omega} + \sqrt{\frac{|A_1|}{N}} \ket{\omega^\bot}.
\end{equation}

维持算子不变,进行运算有:
\begin{equation}
	\begin{cases} 
	G\ket{\omega} = \frac{|A_0| - |A_1|}{N} \ket{\omega} + \frac{2\sqrt{|A_0| \cdot |A_1|}}{N} \ket{\omega^\bot},\\
	G\ket{\omega^\bot} = -\frac{2\sqrt{|A_0| \cdot |A_1|}}{N} \ket{\omega} + \frac{|A_0| - |A_1|}{N} \ket{\omega^\bot}.
	\end{cases}
\end{equation}

整理成矩阵为:
\begin{equation}
\begin{pmatrix}
	\frac{|A_0| - |A_1|}{N} & \frac{2\sqrt{|A_0| \cdot |A_1|}}{N} \\
	-\frac{2\sqrt{|A_0| \cdot |A_1|}}{N} & \frac{|A_0| - |A_1|}{N}
\end{pmatrix}
=
\begin{pmatrix}
	\sqrt{\frac{|A_0|}{N}} &
	-\sqrt{\frac{|A_1|}{N}} \\
	\sqrt{\frac{|A_1|}{N}} &
	\sqrt{\frac{|A_0|}{N}}
\end{pmatrix}^2
=
\begin{pmatrix}
	\cos \theta &- \sin \theta \\
	\sin \theta & \cos \theta
\end{pmatrix} ^2
\end{equation}
其中$\theta = \sin^{-1}\left(\sqrt{\frac{|A_1|}{N}}\right)$.

\subsection{查询算法复杂度下限}
前一节我们得出Grover算法可以通过 $O(\sqrt{N})$ 次查询$U_{x_0}$,以较高的概率找到 $x_0$.事实上,这是渐近最优的,即没有量子算法可以使用少于 $O(\sqrt{N})$ 的访问次数来完成这项任务.

\begin{Quantikz}
	\centering
	\begin{quantikz}
		\lstick{$\ket{0}$}&\gate[6]{V_1}&\gate[3]{U_{x_0}}&\gate[6]{V_2}&\gate[3]{U_{x_0}}&\ \ldots\ &\gate[6]{V_K} &\meter{} \\
		\lstick{$\hspace{-0.7cm}\boldsymbol{\vdots}$} \rstick{$\hspace{5.7cm}\boldsymbol{\vdots} \hspace{2.6cm}\boldsymbol{\vdots}$}\\
		\lstick{$\ket{0}$}&&&&&\ \ldots\ &&\meter{}\\
		\lstick{$\ket{0}$}&&&&&\ \ldots\ &&\\
		\lstick{$\hspace{1.8cm}\boldsymbol{\vdots}\hspace{-2.6cm}\boldsymbol{\vdots}$} \rstick{$\hspace{5.7cm}\boldsymbol{\vdots}\hspace{2.6cm}\boldsymbol{\vdots}$}\\
		\lstick{$\ket{0}$}&&&&&\ \ldots\ &&\\
	\end{quantikz}
	\caption{非结构化搜索算法示意图}
\end{Quantikz}

% 手动重置所有涉及的单元格
\makeatletter
% 清除所有列的宽度状态,假设跨越了 1 到 6 行和多个列
\csundef{cell@width@1-1} \csundef{cell@width@2-1} \csundef{cell@width@3-1} \csundef{cell@width@4-1} \csundef{cell@width@5-1} \csundef{cell@width@6-1}
\csundef{cell@width@1-2} \csundef{cell@width@2-2} \csundef{cell@width@3-2} \csundef{cell@width@4-2} \csundef{cell@width@5-2} \csundef{cell@width@6-2}
\csundef{cell@width@1-3} \csundef{cell@width@2-3} \csundef{cell@width@3-3} \csundef{cell@width@4-3} \csundef{cell@width@5-3} \csundef{cell@width@6-3}
\csundef{cell@width@1-4} \csundef{cell@width@2-4} \csundef{cell@width@3-4} \csundef{cell@width@4-4} \csundef{cell@width@5-4} \csundef{cell@width@6-4}
\csundef{cell@width@1-5} \csundef{cell@width@2-5} \csundef{cell@width@3-5} \csundef{cell@width@4-5} \csundef{cell@width@5-5} \csundef{cell@width@6-5}
\csundef{cell@width@1-6} \csundef{cell@width@2-6} \csundef{cell@width@3-6} \csundef{cell@width@4-6} \csundef{cell@width@5-6} \csundef{cell@width@6-6}
\csundef{cell@width@1-7} \csundef{cell@width@2-7} \csundef{cell@width@3-7} \csundef{cell@width@4-7} \csundef{cell@width@5-7} \csundef{cell@width@6-7}
\makeatother

任何以 $\ket{0}$ 为初始状态并访问$U_{x_0}$~$k$次的量子搜索算法可以被写成以下形式:

\begin{equation}
 \ket{\psi_k^{x_0}} = V_k^{x_0}{\psi_0} = V_k U_{x_0} V_{k-1} U_{x_0} \ldots V_1 U_{x_0} \ket{0} .
\end{equation}

上标 $x_0$ 表示状态依赖于标记量子态 $x_0$,解决这个问题等价于存在 $k$,使得对于每个标记 $x_0$,都有$|\bra{\psi_k^{x_0}\ket{x_0}^2}\geq C$.也就是说,以$\ket{\psi_k^{x_0}}$作为基,测量 $x_0$ 结果为$k$的概率至少为 $1/2$.

为了证明下界,不妨定义一个假算法 $V_k$:~$ \ket{\psi_k} = V_k \ket{0} = V_k V_2 V_1 \ket{0}. $
这个算法不涉及解决方案 $x_0$ 的任何信息,并不能解决搜索问题.因此真实解和假解是可以区分的量子态,也就是说$\|\ket{\psi_k^{x_0}}- \ket{\psi_k}\| \geq C ,~ C>0$.则:
\begin{equation}
\sum_{x_0 \in [n]} \|\ket{\psi_k^{x_0}} - \ket{\psi_k}\| \geq CN.
\end{equation}

特别的有,
\begin{equation}\label{2.18.1}
	\sum_{x_0 \in [n]} \|\ket{\psi_1^{x_0}}- \ket{\psi_1}\| =0.
\end{equation}
这是由于此时还未通过$U_{x_0}$门,另一方面来考虑差的范数:
\begin{equation}\label{2.18}
	\begin{aligned}
	\|\ket{\psi_k^{x_0}} - \ket{\psi_k}\| &= \|V_kU_{x_0}\ket{\psi_{k-1}^{x_0}} - V_k\ket{\psi_{k-1}}\| \\
	&= \|U_{x_0}\ket{\psi_{k-1}^{x_0}} - \ket{\psi_{k-1}}\| \\
	&= \|\ket{\psi_{k-1}^{x_0}} - U_{x_0}\ket{\psi_{k-1}}\| \\
	&= \|\ket{\psi_{k-1}^{x_0}} - \ket{\psi_{k-1}} + \ket{\psi_{k-1}} - U_{x_0}\ket{\psi_{k-1}}\| \\
	&\leq \|\ket{\psi_{k-1}^{x_0}} - \ket{\psi_{k-1}}\| + \|\ket{\psi_{k-1}} - U_{x_0}\ket{\psi_{k-1}}\| \\
    \end{aligned}
\end{equation}

在这里,定义 $\ket{\psi_k} = \sum_{y=1}^n \alpha_{y,t} \ket{y} \ket{\phi_y}$. 也就是专注于第一个系统是量子态$\ket{y}$(即抛掉了下面辅助量子比特),这样可以更清楚的看到$U_{x_0}$对结果的影响. 此时,
\begin{equation*}
U_{x_0} \ket{\psi_k} = \sum_{y \neq x_0} \alpha_{y,t} \ket{y} \ket{\phi_y} - \alpha_{x_0,t} \ket{x_0} \ket{\phi_{x_0}} = \ket{\psi_k} - 2 \alpha_{x_0,t} \ket{x_0} \ket{\phi_{x_0}}.
\end{equation*}
因此
\begin{equation*}
\|U_{x_0} \ket{\psi_k} - \ket{\psi_k}\| = 2|\alpha_{x_0,t}|.
\end{equation*} 
结合\cref{2.18},有:
\begin{equation*}
\|\ket{\psi_k} - \ket{\psi_{k}^{x_0}}\| \leq 2 \sum_{j=1}^{k-1} |\alpha_{x_0,j}|, \quad \forall x \in \{1, \ldots, n\}.
\end{equation*}

\begin{equation*}
\sum_{x=1}^n \|\ket{\psi_k} - \ket{\psi_{k}^{x_0}}\| \leq 2 \sum_{x=1}^n \sum_{j=1}^{k-1} |\alpha_{x,j}| = 2 \sum_{j=1}^{k-1} \sum_{x=1}^n |\alpha_{x,j}|.
\end{equation*}

\begin{equation}\label{2.19.3}
 \leq 2 \sqrt{n} \sum_{j=1}^{k-1} \sqrt{\sum_{x=1}^n |\alpha_{x,j}|^2} = 2(k-1) \sqrt{n}.
\end{equation}

结合\cref{2.18.1},~\cref{2.18},~\cref{2.19.3},可以得出
\begin{equation}
cn \leq 2(k-1) \sqrt{n} \implies k \geq \frac{c \sqrt{n}}{2} + 1 = \Omega(\sqrt{n}).
\end{equation}



		